% LaTeX response letter template

\documentclass[11pt]{article}
\usepackage[utf8]{inputenc}
%\usepackage{lipsum} % to generate some filler text
\usepackage{fullpage}
%\usepackage[letterpaper,margin=1in]{geometry}

% Package needed for optional arguments
\usepackage{xifthen}
\usepackage{amsmath,amsthm,amssymb}
\usepackage{amsfonts,bm,version}
\usepackage{graphicx,fancybox,mathrsfs}
\usepackage{color,xcolor}
\usepackage{mathtools}
\usepackage{enumerate}
\usepackage{caption}
\allowdisplaybreaks

% define counters for reviewers and their points
\newcounter{reviewer}
\setcounter{reviewer}{0}
\newcounter{point}[reviewer]
\setcounter{point}{0}

% Command declarations for reviewer comments and our responses
\newcommand{\reviewersection}{\stepcounter{reviewer} \bigskip \hrule \section*{Reviewer \thereviewer}}

% Comments to the Author
\newenvironment{comments}{\medskip \noindent {\textbf{Comments}}:\ \itshape}{\par}

\newenvironment{point}{\refstepcounter{point} \bigskip \noindent {\textbf{Comment~\arabic{point}}}:\ }{\par }

% Author's Reply
\newenvironment{reply}{\medskip \noindent \textbf{Reply}:\ }{\medskip}

\makeatletter
\def\@maketitle{%
  \newpage
  \null
  \vskip 0.5em%
  \leftline{Reply to the review report on}
  \vskip 1.5em%
  \begin{center}%
  \let \footnote \thanks
    {\Large \bfseries \@title \par}%
    \vskip 1.5em%
    {\large
      \lineskip .5em%
      \begin{tabular}[t]{c}%
        \@author
      \end{tabular}\par}%
    \vskip 1em%
    %{\large \@date}%
  \end{center}%
  \par
  \vskip 1.5em}
\makeatother


% define the new command
\newcommand{\dx}[1][x]{\,{\rm d}#1}
\newcommand{\red}[1]{\textcolor{red}{#1}}
\newcommand{\blue}[1]{\textcolor{blue}{#1}}

\graphicspath{{./figures/}}

% \title{Journal-001: Response to Reviewer Comments}
\title{Journal-001: The full title of the submitted paper}
\author{by \quad \textsc{Author A} ~ and ~ \textsc{Author B}}
%\date{October 27, 2022}

\begin{document}

\maketitle

% title of submitted paper
%{\leftline{Reply to the review report on}}
%\vskip 15pt
%\begin{center}
%{\large\bfseries Journal-001: The full title of the submitted paper }
%
%\vskip15pt
%
%by \quad Author A and Author B
%\end{center}
%\bigskip


%\section*{Response to the reviewers}

% General intro text goes here
The authors thank the reviewers for their critical assessment of our work.
In the following we address their concerns point by point.

% Let's start point-by-point with Reviewer 1
\reviewersection

% Point one description
\begin{comments}
This is comments of Reviewer \thereviewer. With some more words foo bar foo bar ...
\begin{enumerate}
  \item Point one
  \item Point two
  %\item Point three
\end{enumerate}
Suggestion: it's better to test a case with ...
\end{comments}

\vskip 8pt
% Our reply
\begin{reply}
We agree with the reviewer on this important point. This is what we did to fix it.
\begin{enumerate}
  \item Reply one
  \item Reply two
  %\item Reply three
\end{enumerate}
\begin{equation}\label{eq:abc}
  a^2+b^2=c^2.
\end{equation}
Please see Page 5 line 10 of the revised version.
\end{reply}

% Begin a new reviewer section
\reviewersection

% comment 2
\begin{point}
This is comment 1 of Reviewer \thereviewer.
\end{point}

\vskip 5pt
\begin{reply}
Our reply to This is comment 1 of Reviewer \thereviewer.
\end{reply}

% comment 2
\begin{point}
This is comment 2 of Reviewer \thereviewer.
\end{point}

\vskip 5pt
\begin{reply}
Our reply to This is comment 2 of Reviewer \thereviewer.
\end{reply}

% Begin a new reviewer section
\reviewersection

\begin{comments}
This is comments of Reviewer \thereviewer. With some more words foo bar foo bar ...
\end{comments}

\vskip 5pt
\begin{reply}
	Our reply to it with reference to one of our points above using the \LaTeX's
	label/ref system (see also \ref{eq:abc}).
\end{reply}


\end{document}


