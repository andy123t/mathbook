% LaTeX response letter template

\documentclass[11pt]{article}

% Font setting
\usepackage{inputenc}
\usepackage[T1]{fontenc}
%\usepackage{charter}

% Required packages
\usepackage{amsmath,amsthm,amssymb}
\usepackage{mathrsfs}
\usepackage{graphicx}
\usepackage{color,xcolor}
\usepackage{mathtools}
\usepackage{enumitem}
\usepackage{caption}
\usepackage{mathtools}
\usepackage{microtype}
\usepackage{geometry}
\geometry{left=1.25in,right=1.25in,top=1in,bottom=1in}

% allow page breaks between multiline formulas
\allowdisplaybreaks

% command for listsep
\setlist{nolistsep}
\setlist[itemize]{itemsep=3pt}
\setlist[enumerate]{itemsep=3pt}

% command for line spacing
\renewcommand{\baselinestretch}{1.2}

% define counters for reviewers and their points
\newcounter{reviewer}
\setcounter{reviewer}{0}
\newcounter{point}[reviewer]
\setcounter{point}{0}

% package for section etc.
%\usepackage{titlesec}
%\titleformat{\section}{\large\bfseries}{\thesection}{}{}
%\titleformat{\subsection}{\normalfont\bfseries}{\thesubsection}{}{}

% command declarations for reviewer comments and our responses
\newcommand{\reviewersection}{%
  \stepcounter{reviewer}%\bigskip\hrule height 0.5pt%
  \section*{Reviewer~\thereviewer}%
}

% section divider line
\newcommand{\sectionline}{\bigskip\hrule height 0.5pt}

% define color
\definecolor{lightred}{RGB}{216,144,144}
\definecolor{slightred}{RGB}{251,243,243}
\definecolor{slightgreen}{RGB}{243,248,243}

% Referee Comments
\usepackage[most]{tcolorbox}
\tcbuselibrary{breakable,skins}
%\newtcolorbox{commentbox}{
%  blanker, breakable, left=1em,
%  top=0.5em, bottom=0.5em,
%  borderline west={2pt}{0pt}{gray!75!white},
%  colback = {lightred}
%}
\newtcolorbox{commentbox}{
  enhanced, frame hidden, breakable, sharp corners,%
  leftrule=0pt, rightrule=0pt, toprule=0pt, bottomrule=0pt,%
  boxsep=2pt, top=0.5em, bottom=0.5em, left=0.5em, right=0.5em,%
  before skip=1.0em, after skip=1.0em,
  borderline west={2pt}{0pt}{lightred}, %gray!75!white
  colframe=gray, colback = {slightred}
}

\newtcolorbox{changebox}{
  enhanced, frame hidden, breakable, sharp corners,%
  leftrule=0pt, rightrule=0pt, toprule=0pt, bottomrule=0pt,%
  boxsep=2pt, top=0.5em, bottom=0.5em, left=0.5em, right=0.5em,%
  before skip=0.75em, after skip=0.75em,
  colframe=gray, colback = {slightgreen}
}

% define the point environment
\newenvironment{point}[1][Comment]{%
  \refstepcounter{point}%
  \noindent\textbf{#1~\arabic{point}}%
  \begin{commentbox}
  %\color{gray!40!black}
}{%
  \end{commentbox}
}


% define the block environment
\newenvironment{block}[1][]{%
  \noindent\textbf{#1}%
  \begin{commentbox}
  %\color{gray!40!black}
}{%
  \end{commentbox}
}

% package for underlining
\usepackage[normalem]{ulem}
\newcommand{\deleted}[1]{\textcolor{red}{\sout{#1}}}
\newcommand{\added}[1]{\textcolor{blue}{\uwave{#1}}}

% define maketitle environment
\makeatletter
\def\@maketitle{%
  \newpage
  \null
  \vskip 0.5em%
  \leftline{Reply to the review report on}
  \vskip 1.5em%
  \begin{center}%
  \let \footnote \thanks
    {\Large \bfseries \@title \par}%
    \vskip 1.5em%
    {\large
      \lineskip .5em%
      \begin{tabular}[t]{c}%
        \@author
      \end{tabular}\par}%
    \vskip 1em%
    {\large \@date}%
  \end{center}%
  \par
  \vskip 1.0em}
\makeatother

% graphical path
\graphicspath{{./figures/}}

% differential operator
\newcommand{\dif}{\mathop{}\!\mathrm{d}}

% new command
\newcommand{\abs}[1]{\lvert#1\rvert}
\newcommand{\norm}[1]{\lVert#1\rVert}
\newcommand{\dx}[1][x]{\mathop{}\!\mathrm{d}#1}
\newcommand{\ii}{\mathrm{i}\mkern1mu}
\newcommand{\blue}[1]{\textcolor{blue}{#1}}
\newcommand{\red}[1]{\textcolor{red}{#1}}


% information for title and author
\title{Journal-001: The full title of the submitted paper}
\author{by \quad Author A ~ and ~ Author B}
\date{\today}


\begin{document}

\maketitle

% Simple title
%\begin{center}
%{\bfseries\LARGE Response to Reviewers}
%\end{center}
%\section*{Response to Reviewer Comments}

% General intro text goes here
The authors thank the reviewers for their critical assessment of our work.
In the following we address their concerns point by point.

% New reviewer section
\sectionline
\reviewersection

% Comments to the authors
\begin{block}[Comments to the authors]
This is comments of Reviewer \thereviewer. With some more words foo bar foo bar ...
\begin{enumerate}
  \item Point one
  \item Point two
  %\item Point three
\end{enumerate}
Suggestion: it's better to test a case with ...
\end{block}

% Our reply
\noindent\textbf{Author response}
\par\smallskip%

We agree with the reviewer on this point. This is what we did to fix it.
\begin{enumerate}
  \item Reply one
  \item Reply two
  %\item Reply three
\end{enumerate}
\begin{equation}\label{eq:abc}
  a^2+b^2=c^2.
\end{equation}
Please see Page 5 line 10 of the revised version.

\medskip

\noindent\textbf{Changes}

\begin{changebox}
Changes made in the manuscript. Overkill if you also add track changes. To add a track-changes version of the two manuscripts, run \texttt{latexdiff} on the original + revised paper. Goolge latexdiff tutorial to see how that works. Once you do that, you can copy-paste code from the file highlighting difference directly in here. For instance:

\begin{quotation}
Our results indicate that \added{increasing} \deleted{decreasing} the temperature leads to a \added{significant} \deleted{negligible} increase in the growth rate of \added{E. coli} \deleted{bacteria}.
\end{quotation}

\end{changebox}

\clearpage

% New reviewer section
\sectionline
\reviewersection

% comment 1
\begin{point}[Comment]
This is comment 1 of Reviewer \thereviewer.
\end{point}

\noindent\textbf{Reply:} \\[3pt]
Our reply to This is comment 1 of Reviewer \thereviewer.

\medskip

\noindent\textbf{Changes:}

\begin{changebox}
Lots of very important changes
\end{changebox}

\medskip

% comment 2
\begin{point}[Comment]
This is comment 2 of Reviewer \thereviewer. With some more words foo bar foo bar ...
\end{point}

\noindent\textbf{Reply:} \\[3pt]
Our reply to it with reference to one of our points above using the {\LaTeX}'s label/ref system (see also \ref{eq:abc}).

\medskip

\noindent\textbf{Changes:}

\begin{changebox}
So we did not change absolutely anything on this point.
\end{changebox}

% New reviewer section
% \sectionline
% \reviewersection



\end{document}

