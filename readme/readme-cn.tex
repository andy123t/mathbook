% -*- coding: utf-8 -*-
% README LaTeX template

\documentclass[11pt]{article}

\usepackage[UTF8]{ctex}
%\usepackage[english]{babel}
\usepackage{amsmath,amssymb}
\usepackage{graphicx,fancybox,multirow}
\usepackage{mathrsfs}
\usepackage{color,xcolor}
\usepackage{framed}
\usepackage{enumitem}
\usepackage{fontspec}

% package for code highlighting
\usepackage{codestyle}

\usepackage{hyperref}
\hypersetup{colorlinks=true,linkcolor=blue,
filecolor=blue,citecolor=black,urlcolor=cyan,}
\definecolor{winered}{rgb}{0.6,0,0}
\usepackage{caption}

% layout setting
\usepackage{geometry}
\geometry{left=1.25in,right=1.25in,top=1in,bottom=1in}
\setlength{\headheight}{15pt}
\setlength{\headsep}{20pt}

% section setting
\usepackage{titlesec}
\titleformat{\section}{\Large\bfseries}{\thesection}{1em}{}
\titlespacing{\section}{0pt}{2.0ex plus .1ex minus .2ex}{1.5ex plus .1ex minus .2ex}
\titleformat{\subsection}{\large\bfseries}{\thesubsection}{1em}{}
\titlespacing{\subsection}{0pt}{1.5ex plus .1ex minus .2ex}{1.5ex plus .1ex minus .2ex}

% command for line spacing
%\renewcommand{\baselinestretch}{1.35}

% command for listsep
\setlist{nolistsep}

\usepackage{fancyhdr}
\pagestyle{fancy}
\fancyhf{}
\chead{\small\textcolor{gray}{\defheader}}
\cfoot{\small\thepage}

\fancypagestyle{plain}{%
\fancyhf{}
\chead{\small\textcolor{gray}{\defheader}}
\cfoot{\small\thepage}
}

\let\oldheadrule\headrule % Copy \headrule into \oldheadrule
\renewcommand{\headrule}{\color{gray}\oldheadrule}
\renewcommand{\headrulewidth}{0.5pt}

\newcommand{\header}[1]{\def\defheader{#1}}
\newcommand{\version}[1]{\def\defversion{#1}}
% define title page
\makeatletter
\def\@maketitle{%
  \newpage
  \null
  \vspace{-1.5em}%
  \begin{center}%
  \let \footnote \thanks
    {\LARGE \bfseries \@title \par}%
    \vskip 1.0em%
    {\large
      \lineskip .5em%
      \begin{tabular}[t]{c}%
        \@author
      \end{tabular}\par}%
    \vskip 0.5em%
    {\large \defversion}
    \vskip 0.5em%
    {\large \@date}%
  \end{center}%
  \par
  \vskip 1.5em}
\makeatother

% number style of equation
\numberwithin{equation}{section}


% define new command
\newcommand{\red}[1]{{\color{red}#1}}
\newcommand{\blue}[1]{{\color{blue}#1}}
\newcommand{\winered}[1]{\textcolor{winered}{#1}}
\newcommand{\textcode}[1]{\textcolor{winered}{\bfseries\texttt{#1}}~~}

% path of figures
\graphicspath{{./figures/}}

% Information of Document
\header{XX程序文档}

\title{XX程序文档}
\author{作者}
\version{Version 1.x}
\date{2023年10月25日}

\begin{document}

\maketitle

% main body

\section{项目的第一部分}

项目第一部分的简介 ...

\subsection{问题 XX}

\begin{itemize}
\setlength{\itemsep}{3pt}
\item [$\spadesuit$] 考虑模型方程
\begin{equation}\label{eq:NLPossion4}
\left\{
\begin{aligned}
&-\nabla \cdot(a(u) \nabla u) = f,\quad \text{ in } ~\Omega, \\
&~u = 0, \quad \text{ on } ~ \partial \Omega.
\end{aligned}\right.
\end{equation}

\item \textcode{FDM\_Possion.m} 求解 Possion 方程的有限差分法.

\item \textcode{FEM\_Possion.m} 求解 Possion 方程的有限元法.
\end{itemize}


\section{项目的第二部分}

项目第二部分的简介 ...

\begin{enumerate}
\setlength{\itemsep}{3pt}
  \item 第一项
  \item 第二项
  \item 第三项
\end{enumerate}


% examples

\medskip
\noindent $\spadesuit$ 考虑模型方程
\begin{equation}\label{eq:NLPossion4}
\left\{
\begin{aligned}
&-\nabla \cdot(a(u) \nabla u) = f,\quad \text{ in } ~\Omega, \\
&~u = 0, \quad \text{ on } ~ \partial \Omega.
\end{aligned}\right.
\end{equation}


\medskip
\noindent 这是 MATLAB 程序代码高亮环境.

\begin{lstlisting}[style=Matlab,basicstyle=\footnotesize\fontspec{Courier New},title={MATLAB code}]
% Euler method for the ODEs
clear all;  clf
h=0.1; x=0:h:1;
N=length(x)-1;
u(1)=0;
fun=@(t,u) t.^2+t-u;
for n=1:N
    u(n+1)=u(n)+h.*fun(x(n),u(n));
end
ue=-exp(-x)+x.^2-x+1;
plot(x,ue,'b-',x,u,'r+','LineWidth',1)
xlabel('x'), ylabel('u')
\end{lstlisting}

\medskip
\noindent 这是 Python 程序代码高亮环境.

\begin{lstlisting}[style=python,basicstyle=\footnotesize\fontspec{Consolas},title={Python code}]
# Fibonacci series up to n
def fib(n):
    a, b = 0, 1
    while a < n:
        print(a, end=' ')
        a, b = b, a+b
    print()
fib(1000)
\end{lstlisting}



\end{document}

